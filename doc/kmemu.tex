% LTeX: language=en-US
\documentclass[12pt,a4paper]{paper}
\usepackage{a4wide}
\usepackage[margin=2.5cm]{geometry}
\usepackage[czech,english]{babel}
\usepackage{hyphsubst}
\usepackage{graphicx}
\usepackage{indentfirst}
\usepackage{float}
\usepackage{amsmath}
\usepackage{esvect}
\usepackage{listings}
\usepackage{parskip}
\usepackage{tikz}
\usepackage{pgf-umlcd}
\usepackage{underscore}

\selectlanguage{english}
\PassOptionsToPackage{hyphens}{url}

\usepackage[pdfborderstyle={/S/U/W 1}]{hyperref}
\hypersetup{
    colorlinks=true,
    linkcolor=black,
    filecolor=magenta,      
    urlcolor=blue,
    pdftitle={KMEMU - MISC CPU Computer Emulator},
}
\urlstyle{same}

% define helper command for typesetting code
\newcommand{\code}[1]{\texttt{#1}}

\chardef\_=`_

% increase spacing between table rows
\renewcommand{\arraystretch}{1.25}

\begin{document}

\begin{figure}[H]
	\centering
	\includegraphics[width=0.8\textwidth]{pic/kiv-cmyk-en}
\end{figure}

\begin{center}
	\vspace{.5cm}
	\LARGE{KMEMU - MISC CPU Computer Emulator}\\
	\large{Semester project - KIV/PC 2024/25}
\end{center}

\vfill

\noindent
University of West Bohemia in Pilsen \hfill Pavel Altmann - A23B0264P\\
Department of Computer Science \hfill \today
\thispagestyle{empty}

\newpage
\setcounter{page}{1}

\tableofcontents

\newpage

\section{Assignment}
\begin{otherlanguage}{czech}

Naprogramujte v ANSI C přenositelnou \textbf{konzolovou aplikaci}, v podstatě
jednoduchý, ale plně funkční, \textit{emulátor} teoretického počítače – nazvěme
ho třeba K’s Machine (KM) – s procesorem architektury MISC (= Minimal
Instruction Set Computer), který bude na hostitelském počítači vykonávat
programy (v binární podobě, tj. ve strojovém kódu) pro níže popsaný
minimalistický teoretický počítač.

Emulátor se bude spouštět příkazem

\begin{quotation}
\code{X:\textbackslash>kmemu.exe3 \textlangle program.kmx\textrangle
[\textlangle výstupní soubor[.txt]\textrangle]}
\end{quotation}

Symbol \code{\textlangle program ve strojovém kódu\textrangle} zastupuje povinný
parametr – název vstupního binárního souboru s programem ve strojovém kódu pro
procesor KM. Přípona \code{.kmx} (\foreignlanguage{english}{= \underline{K}’s
\underline{M}achine E\underline{x}ecutable}; obdoba windowsovského \code{.exe},
\foreignlanguage{english}{= \underline{Exe}cutable}), označující spustitelný
soubor pro náš teoretický procesor, musí být vždy uvedena. Symbol
\code{\textlangle výstupní soubor\textrangle} pak představuje nepovinný
parametr, kterým je jméno textového souboru, do kterého se bude ukládat výstup
při vykonávání programu, tj. informace/hodnoty, které emulátor vypisuje na
obrazovku při vykonávání instrukcí programu. Není-li druhý, nepovinný, parametr
uveden, směrujte výstup emulátoru do konzole (do standardního streamu
\code{stdout}).

Úkolem Vámi vyvinutého programu tedy bude:

\begin{enumerate}
\item Načíst prvním parametrem určený spustitelný soubor (program) ve strojovém
    kódu ve formátu KMX pro níže popsaný teoretický procesor. \\
\item Vytvořit (podle údajů uložených ve spustitelném souboru) kódový a datový
    segment pro běh programu. \\
\item Vykonat program uložený v kódovém segmentu provedením všech jeho
    instrukcí, včetně smyček, volání podprogramů, atd. \\
\item Provádí-li se právě instrukce, jejímž úkolem je vypsat nějaký údaj na
    obrazovku, vypsat tento údaj do konzole, případně (je-li při spuštění
    programu uveden nepovinný druhý parametr) uložit do specifikovaného
    výstupního textového souboru. \\
\item Po dokončení vykonávání programu uvolnit paměť obsazenou datovým a kódovým
segmentem vykonávaného programu a následně emulátor ukončit. \\

\end{enumerate}

Váš program může být během testování spuštěn například takto (v operačním
systému Windows):


\begin{quotation}
\code{X:\textbackslash>kmemu.exe "E:\textbackslash My Work\textbackslash
Test\textbackslash mersenne.kmx"\ mersenne.txt}
\end{quotation}

Uvažujte všechny možné specifikace (uvedení úplné cesty, relativní cesty, atp.)
jak vstupního, tak výstupního souboru, které jsou přípustné v daném operačním
systému. Pokud nebude na příkazové řádce uveden alespoň jeden parametr (tj.
jméno binárního spustitelného souboru ve formátu KMX), vypište chybové hlášení a
stručný návod k použití programu (v angličtině). Návratová hodnota funkce
\code{int main(...)} bude v tomto případě 1.

\end{otherlanguage}

\section{Problem analysis}

The main problem that permeates the whole assignment is handling of binary data.
Everything from binary input file to addressing both individual bytes and
four-byte integers in the computer's memory. Closely related is going to be the
problem of ensuring no out-of-bounds (OOB) access to the computer's memory.

The solution is to use an array of bytes to represent the computer's memory


The assignment can be divided into two parts: (1) handling binary data in the IO
and the computer's memory and (2) efficiently reading and evaluating
instructions, while ensuring no invalid behavior can occur. This includes mainly
preventing out-of-bounds access to the computer's memory and undefined
arithmetical operations (like division by zero).

Algorithmically the assignment is relatively simple, in its heart it's just a
while loop, that executes the current instruction and moves to the next one,
until either \code{HALT} or the end of file (\code{EOF}) is reached or an error
occurs. The primary source of complexity, in my opinion, is the repetitive code,
which is inherently difficult to debug.

\section{Implementation}

The internal state of the emulated computer is stored in a \code{struct} called
\hyperlink{vm}{\code{VM}} (virtual machine). It holds all relevant values, like
registers, data, code and stack segments (and their sizes), the instruction
pointer (\code{IP}) and few other helper variables.

The program is then segmented into two parts: driver code and VM code. The driver
code handles CLI, IO and calling the relevant VM code, that in turn handles the
execution of the individual instructions itself and the relevant state changes. 

\subsection{Data Structures}

Main data structure of the program is the \hypertarget{vm}{\code{VM}} struct. It's accessed
through interface based on the OOP paradigm with method-like functions
(functions that have a pointer to the \code{struct} as their first parameter).

\begin{center}
\begin{tikzpicture}
    \begin{class}[text width=8cm]{VM}{0,0}
        \attribute{\code{registers} : \code{Registers}}
        \attribute{\code{IP} : \code{size\_t}}
        \attribute{\code{code_segment} : \code{Byte*}}
        \attribute{\code{code_size} : \code{size_t}}
        \attribute{\code{data_segment} : \code{Byte*}}
        \attribute{\code{data_size} : \code{size_t}}
        \attribute{\code{stack_segment} : \code{Byte*}}
        \attribute{\code{stack_size} : \code{size_t}}
        \attribute{\code{error_msg} : \code{char*}}
        \attribute{\code{output} : \code{FILE*}}
        \attribute{\code{flags} : \code{unsigned char}}
        \attribute{\code{instructions\_count} : \code{size_t}}
        \attribute{\code{debug} : \code{DebugLevel}}

        \operation{\code{vm_run(VM *vm)} : \code{int}}
        \operation{\code{vm_print(VM *vm)}}
        \operation{\code{vm_free(VM *vm)}}
    \end{class}
\end{tikzpicture}
\end{center}

\begin{itemize}
    \item \code{registers} - registers stored in a \code{struct} called
        \hyperlink{registers}{\code{Registers}} (below). 
    \item \code{*_segment} - memory segments stored as \code{Byte} arrays, with
        sizes stored in respective \code{*_size} fields.
    \item \code{IP} - instruction pointer, points to the current instruction in
        the \code{code_segment}.
    \item \code{error_msg} - a buffer for error messages, size
        \code{VM_ERROR_BUFFER_SIZE}.
    \item \code{output} - a pointer to the file to which the output should be
        written.
    \item \code{flags} - a bitfield indicating conditions, set by the \code{CMP}
        instruction, read by conditional jumps and zeroed otherwise. For
        possible values, see the \code{FLAG} enum.
    \item \code{instructions_count} - the count of executed instructions, used
        for debugging.
    \item \code{debug} - a flag indicating whether to print debug messages.
\end{itemize}

The \hypertarget{registers}{\code{Registers}} struct is defined as follows 

\begin{center}
\begin{tikzpicture}
    \begin{class}[text width=10cm]{Registers}{0,0}
        \attribute{\code{A} : \code{Word}}
        \attribute{\code{B} : \code{Word}}
        \attribute{\code{C} : \code{Word}}
        \attribute{\code{D} : \code{Word}}
        \attribute{\code{S} : \code{Word}}
        \attribute{\code{SP} : \code{Word}}

        \operation{\code{vm_get_reg(VM *vm, Byte reg, Word *out)} : \code{int}}
        \operation{\code{vm_set_reg(VM *vm, Byte reg, Word value)} : \code{void}}
    \end{class}
\end{tikzpicture}
\end{center}

\textit{Note:} \code{Word} is an alias for \code{signed long int}, a word of
K's Machine, and \code{Byte} is an alias for \code{unsigned char}, one byte.

\subsection{Instructions}

While implementing the individual instructions, great care was given to
eliminating as much repetition as possible. First I tried using helper
functions, but they still required quite a lot of boilerplate for error handling
and propagation. So over time I slowly switched more and more to macros,
eventually allowing me to implement all binary operation instructions each on
single line, only given the name and operator. This resulted in over 2 times
less code than initially, significantly improving readability of the definitions
itself. On the other hand, it comes at the cost of over 100 lines of macro
definitions, making that part of codebase rather complicated and hard to debug.
However, in my opinion it is worth making this tradeoff, since the project
doesn't require long-term support, therefore maintainability is not a major
concern.

\subsection{Data structures}


\subsection{Algorithms}

Similarly to data structures, not many algorithms were needed. The main
algorithm, that is the core of the program, is a simple infinite loop, that each
iteration read a byte from the code segment and gives control to the handler
code of corresponding instruction. 

All instructions operate on a simple basis of loading inputs from the code
segment, executing the operation and storing the result back to either registers
or the data segment of the memory. 

\end{document}
